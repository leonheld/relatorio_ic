\documentclass[12pt, a4paper]{report}
\usepackage{lipsum}

\usepackage{indentfirst} %\section or \chapter by default don't indent the first paragraph for some absurd reason!

\begin{document}
\begin{titlepage}
    \begin{center}
        \vspace*{1cm}
        \Huge
        \textbf{Título}

        \vspace{0.5cm}
         Subtítulo
             
        \vspace{1.5cm}
        \LARGE
        Relatório Final de \\
        pesquisa de Iniciação Científica
 
        \vfill
        \Large
        \textbf{Aluno: Leonardo José Held}\\
        \textbf{Orientador: Raimes Moraes}
        \vspace{0.8cm}

        Departamento de Engenharia Elétrica e Eletrônica\\
        Curso de Graduação em Engenharia Eletrônica\\
        Universidade Federal de Santa Catarina\\
        Brasil\\
        %TODO:
        
             
    \end{center}
 \end{titlepage}
 \Large
\chapter[]{Resumo}
\lipsum[10]

\chapter[]{Introdução}

Revisão bibliográfica: Sistemas médicos, sistemas embarcados, sensores,
comunicação embarcada, sistemas em tempo real. \\
Livros que podem ajudam: Sistemas Operacionais, Manuais da TI, Making Embedded Systems e 
Understanding Linux Kernel.\\

Justificativa: meme\\

Objetivos: Adicionar às pesquisas de processamento de sinais
do laboratório e fazer uma interface similar à estetoscópios só que barato

\chapter[]{MATERIAL E MÉTODOS}
Falar um pouco sobre as plaquinhas usadas, software utilizado, como os sensores são feitos pra encaixar etc e a resposta que se deve esperar deles.

\chapter[]{RESULTADOS E DISCUSSÃO}
Avaliar a melhora do sistema e o que eu aprendi

\chapter[]{CONCLUSÕES}
O que deve ser feito após isso

\chapter[]{REFERÊNCIAS}
Listar todas as referências que eu utilizei
\end{document}